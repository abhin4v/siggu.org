\documentclass{article}

\usepackage{polyglossia}
\setdefaultlanguage{hindi}
\setotherlanguage{sanskrit}
\usepackage{fontspec}
\newfontfamily\devanagarifont[Scale=MatchUppercase]{Kohinoor Devanagari}

\usepackage{graphicx}
\usepackage{subcaption}
\usepackage{float}
\usepackage[bottom]{footmisc}
\usepackage{enumitem}
\usepackage{hyperref}

\setlength{\parindent}{1.3em}
\setlength{\parskip}{0.7em}
\setlist{noitemsep}

\begin{document}

प्रिय विपाशयना केंद्र सेवकों,

पिछले तीन सालों में, मैं बहुत भाग्यशाली रहा कि आप के विपाशयना केंद्र में, आप लोग की मुझे कई बार ध्यान करने का मौक़ा मिला साथ
ही कई बार सेवा करने का मौक़ा मिला। जब हम लोग सामान्य परंतु महत्यपूर्ण: सब्ज़ियाँ काटना, रोटियाँ
बनाना, बर्तन धोना, कमरे साफ़ करना, कपड़े धोना, नई इमारतें बनाना, और दीवारें रंगना। जब कभी मैं
लोगों से इन कामों के बारे में बात करता हूँ, ये समझना बहुत मुश्किल होता है कि ये काम इतने ज़रूरी क्यों हैं।
मैं कहानी के माध्यम से समझाने की कोशिश करूँगा।


\begin{center}
\line(1,0){250}
\end{center}

2012 में, मैं भारत आया। मैं अमीर था पर बहुत दुखी। मैं इतना अमीर था कि में दुनिया कहीं भी जा सकता
था, जहाँ मेरा मन करें, कोई भे कार ख़रीद सकता था, किसी भी रेस्तराँ में खा सकता था ... पर मुझे कभी
नहीं लगता था कि ये काफ़ी है। मैं हमेशा ज़्यादा की चाह रखता था।

उस समय मैं बहुत शराब पीता था। मैं बहुत धूम्रपान करता था, नशीली चीज़ें लेता था। मैं अजीव घंटो में सोता
था। मैं एक ग़ैरज़िम्मेदार कर्मचारी था --- मैं काम पर देरी से पहुँचता था और अकसार मैं अपन्हे आलस के कारण
काम पर जाता ही नहीं था। मैं बहुत अस्वस्थकर खाना खाता था और मैं कभी कभार ही कसरत करता था। मैं
बहुत बीमार पड़ता था।

मैं शारीरिक और मनिसिक रूप से टूट गया था।

क्योंकि मेरा मन इतना बीमार था, मैं ग़लत निर्णय ले रहा था। मैं अपने दोस्तों और सहकरमचारियों को बहुत
दुखदाई बातें कहता था और उनपर चिल्लाता भी था। मैं हर समय ग़ुस्सा रहेता था। मैं इतना ग़ुस्सा रहेता था
कि साँस लेने में भी दर्द होता था।




\end{document}
