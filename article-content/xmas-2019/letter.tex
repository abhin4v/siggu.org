\documentclass{article}

\usepackage{polyglossia}
\setdefaultlanguage{english}
\setotherlanguage{sanskrit}
\usepackage{fontspec}
\newfontfamily\devanagarifont[Scale=MatchUppercase]{Kohinoor Devanagari}

\usepackage{graphicx}
\usepackage{subcaption}
\usepackage{float}
\usepackage[bottom]{footmisc}
\usepackage{enumitem}
\usepackage{hyperref}

\setlength{\parindent}{1.3em}
\setlength{\parskip}{0.7em}
\setlist{noitemsep}

\begin{document}

I'm looking back over Preethi's blog (which has evolved into a semi-regular ``month
note'' from her previous attempt at ``week notes'') and noticing that she started
writing right around Christmas of last year. This makes her blog a great resource for
me to pull data from... a mine of memories for the past twelve months. If you want to
read her memoires, here blog is here: \url{https://medium.com/@preethig}

Collectively, we've been writing things on \url{https://www.siggu.org}

2019 followed an arc: from Canada for Christmas, to Chennai for the new year of 2018/2019,
to Jaipur for an inspiring Literature Festival, to Kothagiri, Bangalore,
Sikkim. From India we traveled through Ireland to Canada for summer 2019, spending
most of the time in the Nova Scotia cabin. After that, we met my new niece in
Saskatchewan and returned to India.

So, let's rewind back to the new year of 12 months ago to see where this year has gone.

Around Christmas of last year, we both finished Yuval Noah Harari's third book:
\textit{21 Lessons for the 21st Century} and enjoyed it greatly. I've since given the
book to a large number of people... and accidentally gave it to Preethi's father
twice. It's a quicker and easier read than either \textit{Sapiens} or \textit{Homo
  Deus} and, despite having ``Lessons'' in the title, it comes off as less
instructive. It feels like everyone --- the world over --- is sick and tired of the
state of humanity right now. Taking a breather from Facebook and Twitter is always a
good idea but these days it's hard to turn in any direction without being assaulted
with the peculiar, unbalanced news of the world. \textit{21 Lessons} has some really
helpful ideas for maintaining one's sanity. We both read Hans Rosling's
\textit{Factfulness} around the same time and found it to be useful in a similar
way.

After a Christmas full of board games we spent a bit of time in Calgary and Vancouver
visiting Conrad and Franca, then Geoff and Dana. The fall of 2018 was spent in a whirlwind
tour of Canada. Preethi is
considering a PhD in Canada and spending a hot minute in each of the major cities
seemed like a good way to get a feel for which ones she might want to live in. She's
a lot more enthusiastic about Winnipeg than I am. New Years 2018/2019 also marked the end
of six months of experimental homelessness for
us. Turns out... homelessness mostly sucks. Who would have thought? Jumping from
place to place, a person starts to
really long for a proper home to put down some roots. But we'd planned to split the
year between Canada and India and so we continued on our path toward India to do some
more exploring.

In early 2019, we spent a long time in Chennai (and we're back there now, in fact) visiting
Preethi's parents, sister, and grandma. When we landed in Chennai we noticed that the city had
put up a number of cute pro-bicycle roadsigns. Some of them were informational but
quite a few of them said things like ``Happiness is My Bicycle And A Sunny Day'' or
``0\% Pollution, 100\% Emotion'' ...rather endearing in a city that regularly gets to
35 degrees celcius with stifling humidity. It's midday on December 30th as I write
this and Chennai is 28 degrees. This is ``winter''. I soaked my shirt in sweat just walking to the cafe
I'm in... so you can imagine what riding a bicycle here is like the rest of the year.

After Chennai and our visit to Jaipur for the Literature Festival, we spent a while
in Bangalore. We've both been half-heartedly running a little Ayurvedic medicine
experiment on ourselves with my old Ayurvedic doctor in Bangalore. He's a curious guy
with a deep, penetrating stare and a strong conviction that radishes and eggplants
are no better than rat poison. I'm with him on the eggplant... tastes fine, but the
stuff is basically as nutritious as styrofoam. Part of his treatment involves getting ghee (clarified
butter) shot up your butt, every day for a week, in the form of a Butt Butter Enema.

Preethi was not impressed.

We also visited Preethi's real doctor (a pulmonologist) while we were in
Bangalore. He's a grouchy old army doctor who yelled at her the first time she
visited him: ``You used to smoke! Your lungs are overinflated! Are you
\textit{trying} to kill yourself?!'' Once she got past his brand of Pulmonology
Initiation, he was really quite a sweet fellow and Preethi's lung status has
graduated from ``Jesus Murphy, how are you even alive?'' to ``You can run a half marathon? Good job.''

We've been running regularly as a way, initially, of increasing Preethi's lung
capacity. But we've both found it to be a good excuse to listen to some trashy upbeat
gym music for an hour every other morning. Team favourites include ``What's Up
Danger'' from the \textit{Into The Spiderverse} (best movie of the year, by the way)
soundtrack and Janelle Monae's ``Django Jane''. For three minutes every other morning
I pretend I'm a really cool black woman instead of a really lame white dude.

On our way from Bangalore to Chennai we thought it would be a fun to challenge
ourselves by riding our bikes all the way, 100 kilometres at a time. It's about a
400km journey with plenty of places to stop and eat or get a fresh coconut water and
we packed ourselves about 10 litres of water, just in case. We did not think to pack
sunscreen. Somewhere around lunch on the day of our first 100km leg, we both realized our
mistake. I've never had a sunburn before that made my \textit{muscles} ache. Preethi
has just... never had a sunburn before. We arrived in
Krishnagiri, our first of three stop points, mid-afternoon on our first day. We
thought we were going to die. My sunburn lasted months. Preethi's turned into a tan
within a few weeks. Note to self: Tamil Nadu gets a lot of sun once you escape the
city's protective air pollution shield.

After our failed bicycle trip, we each took a Vipassana course. This was my first
20-day course and Preethi's fourth 10-day course. Sitting in silence for 20 days is
certainly more difficult than for 10 days but because new students are not allowed to
take 20-day courses it was a welcome relief from the usual noise (read: burps and
farts) that accompanies the
10-day courses. The meditation courses were a good precursor to our trip to Sikkim,
one of our most-anticipated destinations for the year.

I had wanted to visit Sikkim after reading John Coleman's book \textit{The Quiet Mind}
years earlier. The book about his life as a CIA agent, starting in Thailand. He found the Thai people
fascinating. Their patience and tolerance was almost otherworldly to him and over the
years his curiosity grew until it reached a head and he started experimenting with
different meditation techniques. At the point in the book where his experiments take
him to Sikkim, he describes it as ``the most beautiful place in the world'' and
suggests that a person should make ever effort to visit Sikkim if the borders ever
open up. Now that Sikkim is a part of India, it is possible (although it requires a
bit of annoying paperwork) to visit. It's really unlike any other part of India. It
is much more like a mashup of Bhutan, Tibet, and Nepal --- the countries it borders
--- than it is like any other state in India. It's also a subtropical
rainforest... in the mountains. So the climate is pretty peculiar. It's raining
constantly, cool enough to feel comfortable to a Canadian but warm enough to feel
comfortable to a Tamilian. I think we both really enjoyed this peculiar Indian
state.

...but our experience there wasn't really notable because of the city of Gangtok or
the strange culture or the beautiful green hillsides. What was most notable were the
people we met while we were there. We had intended to spend time in Sikkim writing
and exploring and perhaps visiting the local Vipassana Centre to see what it was
like. We contacted the Vipassana Centre with the phone number from Google Maps and
the man on the other end said he would happily drive us to the Centre from
Gangtok. The next morning we jumped into an SUV and spent a bumpy hour riding up the
mountain from the city to the Vipassana Centre with these three men. They spoke in
Bhutia most of the way and at first we really felt like we were imposing on
them. However, after a couple of days at the Centre we found that these three men
more or less managed the entire Centre alone. Not only were they responsible for
running monthly 10-day Vipassana courses but they were also supervising the
construction of a massive concrete pagoda for meditation within the centre. It was
really quite a beautiful thing to see people willing to invest so much of their time
simply helping other people, expecting nothing in return. But our biggest surprise
was yet to come.

With these three fellows we worked in the Vipassana Centre, mostly cleaning and
organizing, for our first two weeks in Sikkim. At the end of these two weeks we
decided to continue volunteering for the rest of the month, serving a 10-day
Vipassana course that was about to start. We found out the course would be conducted
by the senior teacher in the area --- a woman in her 90s.

At first, we had serious concerns about the teacher. One has no doubt about such a
senior teacher's ability to give the course instruction... but how would a
92-year-old woman manage such a strenuous schedule? From 4:00 AM to 10:00 PM? As it
turns out, we need not have worried. She ate the spiciest food at the centre. She
would sit and answer students' questions for \textit{hours}. By the end of the
course, on Day 7, 8, and 9 when the course is getting very difficult for students but
also very serious, she would spend the entire day in the meditation hall. All day she
would call on students, one by one, and hold a long interview with them to make sure
each and every one of them understood the technique. These interviews were conducted
in English, Bhutia, Hindi, Nepali, and Bengali because, despite having no formal education,
she was fluent in seven languages. One doubt remained for us, though: She kept
rubbing her arm every hour or two and I wondered if she was really healthy enough to
run the course.

After the course was over, we found out that she had been run over by a car the year
prior. The car had shattered her arm. It was little wonder she was rubbing it now and
then... it was more of a wonder how she was alive at all. Over the years, I've
met many strong women and many testaments to the benefits of Vipassana meditation,
but this lady takes the cake on both counts.

I don't think either of us was ready to leave Sikkim when we did but we had to return
to South India to say goodbye to Preethi's family before heading to Canada for the
summer. We spent a weekend in Kothagiri at Preethi's aunt's house in the hills... a
place where we might spend some extended time to write in seclusion, come February 2020.

After Sikkim, Kothagiri, and a brief stop in Ireland, we headed on to the fortyfourforty coop cabin in Nova
Scotia. Over the years, the property there has grown to include a well, a wood stove,
a toilet, and solar panels. It's still rustic but it's liveable and we were curious
just how liveable it had become. We decided to test this out by living there for four
months. We bought some maintenance-free bicycles from Halifax and began our foray
into our first period of semi-seclusion. Preethi planted a formidable garden but
someone kept coming in the night and eating the greens off the beats and the
carrots. We found out eventually that it was the resident porcupine, who we caught
leaning up on the pea fence and snagging peas one evening. She and her baby would
usually be out by the beach every evening, watching the sunset with us, so it was hard to get too
mad at them.

Our summer in the cabin went by quickly. We installed floor insulation, built a wood
box to handle our chopped logs, made a coat tree on Preethi's birthday, and --- most
significantly --- built a tool shed with Mom and Dad to house our canoe and bikes and assorted junk
during the off-season. For my birthday, Preethi and I ran our first half-marathon in
the Lobster Capital of Canada, Barrington.

After four quick months in the cabin, we headed west to Saskatchewan to say hello to
my very new niece, Kaia. Quite an adorable kid, she spent the entire time with us
smiling and giggling. She looks just like Aaron did as a baby. And on the rare
occasions when she'd poop or cry Mom or Dad
would step in to spend time with her! Being an uncle is great.

Now we're back in India. Preethi is finishing the data science course she's spent the
past six months working on with John Hopkin's University through Coursera. Now that the course
is wrapping up she's volunteering with Civic Data Lab, an
organization that works with national open data in India. We're both attempting to
construct PhD proposals... Preethi's in epidemiology and public health, mine in the
much farther afield, and mostly uncharted, discipline of meditation research.

2020 will see us completing these PhD proposals and hopefully finding a city in which
we might pursue them. After that... god knows what. Guess we'll go back to being poor
University students again for a few years. Ha ha! Ugh.

Preethi closed out most of her Month Notes with a list of books she read that month
but I think we'll include our book recommendations for the year here, instead.

As I mentioned in the beginning, \textit{21 Lessons for the 21st Century} and
\textit{Factfulness} are both worth your time if you're into non-fiction. We both
read the English translation of Mahasweta Devi's completely insane \textit{Romtha}
and thought it was brilliant. We both read and enjoyed Naomi Alderman's \textit{The
  Power}. And in the field of neuroscience there's \textit{Altered Traits} and
Michael Pollan's \textit{How to Change Your Mind: The New Science of Psychedelics}.

We hope everyone had a wonderful Christmas with friends and family and is excited
about all the new opportunities of 2020!

Thinking about you fondly and often.

Love,
-steven \& preethi

\end{document}
